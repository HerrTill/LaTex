% ---- Präambel mit Angaben zum Dokument
\input{Inhalt/00_Latex/praeambel}

% ---- Elektronische Version oder Gedruckte Version?
% ---- Unterschied: Die elektronische Version enthält keinen Platzhalter für die Unterschrift
\usepackage{ifthen}
\newboolean{e-Abgabe}
\setboolean{e-Abgabe}{false}    % false=gedruckte Fassung

% ---- Persönlichen Daten:
\newcommand{\titel}{Contract-based testing in Kubernetes Custom Resource}
\newcommand{\arbeit}{Bachelor Thesis}
\newcommand{\studiengang}{Internationale Wirtschaftsinformatik}
\newcommand{\autor}{Till Müller}
\newcommand{\autorReverse}{Müller, Till}
\newcommand{\verfassungsort}{Mudau}
\newcommand{\matrikelnr}{633641}
\newcommand{\kurs}{IBAIT20}
\newcommand{\abgabe}{07.06.2023}
\newcommand{\firmaName}{SAP SE}
\newcommand{\firmaStrasse}{Dietmar-Hopp-Allee 16}
\newcommand{\firmaPlz}{69190 Walldorf, Deutschland}
\newcommand{\betreuerFirma}{Johannes Häussler}
\newcommand{\betreuerHWG}{LB Patrick Gutgesell}

% ---- Metainformation für das PDF Dokument
\hypersetup{
	pdftitle    = {\titel},
	pdfsubject  = {\arbeit},
	pdfauthor   = {\autor},
	pdfcreator  = {LaTeX},
}

% ---- Definition der Kopf- und Fußzeilen
\clearpairofpagestyles                          % Löschen von LaTeX Standard
\automark[section]{chapter}                     % Füllen von section und chapter
\renewcommand*{\chaptermarkformat}{}            % Entfernt die Kapitelnummer
\renewcommand*{\sectionmarkformat}{}            % Entfernt die Sectionnummer

\ifoot[]{}                                      % Fußzeile links
\cfoot*{\sffamily\pagemark}                     % Fußzeile mitte
\ofoot[]{}                                      % Fußzeile rechts
\KOMAoptions{
   headsepline = 0.2pt,                         % Liniendicke Kopfzeile
   footsepline = false                          % Liniendicke Fußzeile
}


% ---- Hilfreiches
\newcommand{\zB}{z.\,B. }   % "z.B." mit kleinem Leeraum dazwischen (ohne wäre nicht korrekt)
\newcommand{\dash}{d.\,h. }

\newcommand{\code}[1]{\texttt{#1}} % Ist einfacher zu schreiben als ständig \texttt und erlaubt
                                   % Änderungen im Nachhinein, wenn man z.B. Inline-Code anders stylen möchte.

% ---- Silbentrennung (falls LaTeX defaults falsch / nicht gewünscht sind)
\hyphenation{HANA}         % anstatt HA-NA
\hyphenation{Graph-Script} % anstatt GraphS-cript

% ---- Beginn des Dokuments
\begin{document}
\setlength{\parindent}{0pt}              % Keine Paragraphen Einrückung.
                                         % Dafür haben wir den Abstand zwischen den Paragraphen.
\setcounter{secnumdepth}{2}              % Nummerierungstiefe fürs Inhaltsverzeichnis
                 % Tiefe des Inhaltsverzeichnisses. Ggf. so anpassen,
                                         % dass das Verzeichnis auf eine Seite passt.
\sffamily                                % Serifenlose Schrift verwenden.

% ---- Vorspann
% ------ Titelseite
\singlespacing
\thispagestyle{empty}
\begin{titlepage}
\enlargethispage{4cm}

\begin{figure}           % Logo vom Ausbildungsbetrieb und der DHBW
	\vspace*{-5mm} % Sollte dein Titel zu lang werden, kannst du mit diesem "Hack" 
	%                  den Inhalt der Seite nach oben schieben.
	\begin{minipage}{0.49\textwidth}
		\flushleft
		\includegraphics[height=2.6cm]{Bilder/Logos/Logo_SAP.pdf} 
	\end{minipage}
	\hfill
	\begin{minipage}{0.49\textwidth}
		
		\includegraphics[height=2cm]{Bilder/Logos/Logo_HWGLU.png} 
	\end{minipage}
	\hfill
\end{figure} 
\vspace*{0.1cm}

\begin{center}
	\huge{\textbf{\titel}}\\[1.5cm]
	\Large{\textbf{\arbeit}}\\[1cm]
	\normalsize{Part of the Examination for the \\[0.2cm]Bachelor of Science (B.Sc.)\\[0.2cm] of \\[0.2cm]}
	\normalsize{International Business Administration and Information Technology}\\[0.2cm]
	\normalsize{at the University of Business and Society Ludwigshafen}\\[2cm]
\end{center}

\begin{center}
	\vfill
	\normalsize{by}\\[0.5cm]
	\begin{tabular}{ll}
		Till Thomas Müller	\\[0.2cm]
		Gartenweg 28  		\\[0.2cm]
		69427 Mudau			\\[2cm]
	\end{tabular} 
\end{center}

\begin{center}
	\vfill
	\begin{tabular}{ll}
		Date of submission:		& \abgabe \\[0.2cm]
		Company Supervisor:		& \betreuerFirma \\[0.2cm]
		Academic Supervisor:	& \betreuerHWG \\[2cm]
	\end{tabular} 
\end{center}
\end{titlepage}
  % Titelseite
\newcounter{savepage}
\pagenumbering{Roman}                    % Römische Seitenzahlen
\onehalfspacing

% ------ Erklärung, Sperrvermerk, Abstact
%\include{Inhalt/01_Standard/erklaerung}
%\include{Inhalt/01_Standard/sperrvermerk}
%\include{Inhalt/02_Abstract/abstract-en}
%\include{Inhalt/02_Abstract/abstract-de}

% ------ Inhaltsverzeichnis
\singlespacing
\setcounter{tocdepth}{10}
\tableofcontents

% ------ Verzeichnisse
\renewcommand*{\chapterpagestyle}{plain}
\pagestyle{plain}
\begin{spacing}{0.5}
\chapter*{List of Abbreviations}
\addcontentsline{toc}{chapter}{List of Abbreviations} % Hinzufügen zum Inhaltsverzeichnis 

\begin{acronym}[WYSISWG] % längstes Kürzel wird verw. für den Abstand zw. Kürzel u. Text

	% Alphabetisch selbst sortieren - nicht verwendete Kürzel rausnehmen!
	\acro{AI}{Artificial Intelligence}

\end{acronym}
\end{spacing}
\listoffigures                          % Erzeugen des Abbildungsverzeichnisses 
\listoftables                           % Erzeugen des Tabellenverzeichnisses
\lstlistoflistings                      % Erzeugen des Listenverzeichnisses


% ---- Inhalt der Arbeit
\cleardoublepage
\pagenumbering{arabic}                  % Arabische Seitenzahlen für den Hauptteil
\setlength{\parskip}{0.5\baselineskip}  % Abstand zwischen Absätzen
\rmfamily
\renewcommand*{\chapterpagestyle}{scrheadings}
\pagestyle{scrheadings}
\onehalfspacing

\include{Inhalt/04_Inhalt/introduction}
\include{Inhalt/04_Inhalt/ObjectivesandCriteria}
\include{Inhalt/04_Inhalt/fundamentals}

\include{Inhalt/04_Inhalt/content_dataset_selection}
\include{Inhalt/04_Inhalt/content_business_understanding}
\include{Inhalt/04_Inhalt/content_data_understanding}
\include{Inhalt/04_Inhalt/content_data_preparation}
\include{Inhalt/04_Inhalt/content_modeling}
\include{Inhalt/04_Inhalt/content_evaluation}
\include{Inhalt/04_Inhalt/content_deployment}

\include{Inhalt/04_Inhalt/Evaluation}
\include{Inhalt/04_Inhalt/Outlook}
% ---- Literaturverzeichnis
\cleardoublepage
\renewcommand*{\chapterpagestyle}{plain}
\pagestyle{plain}
\pagenumbering{Roman}                   % Römische Seitenzahlen
\setcounter{page}{\numexpr\value{savepage}+1}
\printbibliography[title=References]

% ---- Anhang
\appendix
\include{Inhalt/04_Inhalt/appendix}
\newpage
\end{document}
