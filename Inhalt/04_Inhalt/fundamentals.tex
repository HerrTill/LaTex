\chapter{Fundamentals}
\section{Glossary of Terms}

	\subparagraph{Clustering Algorithm} is a sequence of instructions, which arranges a set of instances into groups, which contain items of high similarity to each other.
	\subparagraph{\ac{NLP}}is often attributed to computer science, but after closer examination, \ac{NLP} is a discipline comprised of linguistics, computer science, artificial intelligence an mathematics \cite{chowdhury2003}.
	\subparagraph{Data Cleaning} 
	\subparagraph{Data Wrangling}
	\subparagraph{Feature Extraction}

	\begin{itemize}
		\item SAP SE
		\item SAP Leonardo MLF
		\item SAP AI Core
		\item SAP AI Launchpad
		\item AI Foundation
	\end{itemize}

\section{Corporate Environment}

	\subsection{Historical}
	SAP was founded in 1972 by five ex-IBM employees. The original company name was "Systemanalyse Programmentwicklung", which can be translated to "System analysis and program development". In 1976, a second company, the SAP GmbH was founded, where the acronym SAP denoted "Systems, Applications and Products for data processing" \cite{GeschichteSAP1972}. The SAP GmbH is the company, which is today known as SAP SE. 
	
	Data processing being part of the company's name shows the importance of this field to SAP since the beginning of the company history.
	In 2017 SAP entered the AI business with the SAP Leonardo Machine Learning Foundation \cite{rutschmannSAPLeonardo2021}, and challenged the market for hyped products in machine learning, blockchain, big data and design thinking. The name Leonardo refers to Leonardo Da Vinci, who is renowned for his interdisciplinary innovations \cite{schmitzLeonardo}. SAP has the goal of driving the digital innovation strategies of customers with the help of SAP Leonardo.
	
	
	With changes in the underlying hyperscalers for Leonardo, and evolving requirements of customers and partners, SAP adjusted their AI strategy. Two new products were introduced: SAP AI Core and SAP AI Launchpad. Both products are united under the collective name of AI Foundation \cite{rutschmannSAPLeonardo2021}. With the general availability of AI Foundation in late 2021, SAP Leonardo is officially sunsetted.
	
	
	
	- What comes in the future?
	
	- how much does SAP earn with AI?
	
	\subsection{Organizational}
	SAP SE has an executive board consisting of seven members, each attributed to one area. The \ac{AI} division falls into the responsibility of Jürgen Müller, Chief Technology Officer and leader of the board area for technology and innovation \cite{JuergenMuellerBiography}.
	Members of the organizational unit for \ac{AI} are divided in different teams concerned with development, product success, operations and specific \ac{AI}-services. Development Teams are organized into \acp{CoE}, in which special expertise for designated areas is united.
	SAP has a team of researchers around the world concerned with state-of-the-art topics, including few-shot learning, sentiment analysis, privacy and fairness \cite{AIOverviewResearch}. The research teams regularly publish articles on their advancements.
	
	\subsection{Technological}
	https://www.sap.com/products/artificial-intelligence.html

\section{Machine Learning}
Already Alan Turing understood that for laymen a learning machine can be perceived as a paradox.  How can a machine learn, if a human has to define its behavior beforehand? There are three major subfields in the discipline of artificial intelligence that fundamentally explain how a computer can learn how to behave despite predefined behavior.

\subsection{Supervised Learning}
A supervised learning algorithm learns its decision with the help of a data set (input) that also contains the correct decision (output) as information. It is trained with only a part of the entire data set, so that the model can be tested in a later step with the help of unknown data. This way, a statement can be made about the accuracy of the model.

\subsection{Unupervised Learning}
Unsupervised learning is complementary to supervised learning. All algorithms that fall into the category of unsupervised learning are trained with data that does not contain the correct output (label) as information. Here, the categorization is not constrained by the given data, but decided on by the algorithm.

\subsection{Reinforcement Learning}
The third way in which an algorithm can make better decisions as it gains experience is called reinforcement learning. Reinforcement learning is about letting algorithms solve very complex tasks. The special feature is that there is no defined solution path, but the algorithm is rewarded for goal-oriented behavior and punished for wrong decisions. The definition of goal-oriented behavior has to be put into place by the engineers setting up the training of the model. Real-world tasks are extremely complex, so not all possible solution paths can be calculated and compared to find the optimal path. Parking a car is a routine task for a human after a few hours of driving, but a computer sees only an infinite set of possibilities for turning angles. This problem can be solved by reinforcement learning. The algorithm is rewarded for each parking attempt where the car ends up seeing in the parking space. For the remaining attempts, the algorithm is penalized. Over many thousands of attempts, the reinforcement learning model is trained in this way.

The three major ways of learning even with previously defined behavior can now be implemented by specific models. For example, there are several ways to create and train a model using Unsupervised Learning.

\subsection{Clustering Algorithms}
multinomial, one bad example for a clustering would be the closest 5 docs to each one (this is multilabel)

\section{SAP AI Core}
\subsection{Docker}
